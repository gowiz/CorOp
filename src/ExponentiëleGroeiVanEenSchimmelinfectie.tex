\documentclass[12pt]{article}

\input{/home/ppareit/kaa1920/pitex.tex}

\begin{document}

Een schimmelinfectie groeit exponentiële, na onderzoek blijkt dat de oppervlakte van de schimmelinfectie gegeven wordt door
$$N_t=260\cdot 1.5^t$$
met $t$ in uur en $N_t$ in $\cm^2$.
\begin{enumerate}[(a)]
  \item Bepaal de beginwaarde $N_0$
    \arules{1}
  \item Bepaal het groeipercentage $p$
    \arules{1}
  %\item Gaat het hier om lineaire of exponentiële groei? Geef de bijhorende groeiterm/groeifactor per jaar.
  \item Bereken de waarden van $N_t$ na 6 uur.
    \arules{1}
  \item Hoe lang duurt het voordat de schimmelinfectie een oppervlakte heeft van $1000 \cm^2$.
    \arules{1}
  \end{enumerate}
  \vspace{1cm}
In bovenstaande oefening kunnen volgende formules handig zijn:
$$\log_a{x} = y \LRA a^y=x$$
$$\log_a{x}= \dfrac{\log x}{\log a}$$

\end{document}
